%!TEX root = ../thesis.tex
%*******************************************************************************
%****************************** Third Chapter **********************************
%*******************************************************************************
\chapter{Neural Tensor Factorisation}

% **************************** Define Graphics Path **************************

\ifpdf
    \graphicspath{{Chapter3/Figs/Raster/}{Chapter3/Figs/PDF/}{Chapter3/Figs/}}
\else
    \graphicspath{{Chapter3/Figs/Vector/}{Chapter3/Figs/}}
\fi


Open domain question answering is an area of research concerned with reasoning about knowledge expressed in natural language. Knowledge base (KB) link prediction using latent feature modelling is an approach that has been applied to this task, with varying degrees of success  ~\citep{nguyen2017novel, diefenbach2018wdaqua, kristiadi2019incorporating}. This approach aims to rank plausible relationships between entities by using using trainable embedding representaitons. Previous latent feature models have focused on limiting the number of parameters so as to be able to scale to large KBs, at a cost of complex entity-relational interaction modelling. \newline
Neural tensor networks (NTN), An extension to tensor factorisation was the first successful approach at taking advantage of expressiveness of deep models. It relied on adding a recursive network (RCN) representation to the bilnear tensor product, and wrapping that representation in a fully connected layer to then compute a relational score. Convolutional entity representations (ConvE) were similarly effective at extending tensor factorisation approaches by applying the convolutional operation in modelling entity-relational interactions, before completing a factorisation to compute a relational score. Hypernetworks were then applied to the ConvE model (HypER) to add further expressive power by pre-computing relational filter representations for subsequent use in a neural tensor factorisation. \newline
In this chapter we examine the NTN model, and attempt a simple improvement by applying Adam optimisation and hyperparameter random search to the training algorithm. We then introduce HypER+, an extension to the HypER model which compensates covariate shift introduced by augmenting relational filters with a hypernetwork. We then extend HypER+ to make use of pre-trained GloVe word embeddings, and address the problem of entity interaction sparsity in KBs.


% **************************** Section 1 **************************

\section{Neural Tensor Networks}

\subsection{Neural Tensor Factorisation}
RESCAL introduced the bilinear tensor product for relational scoring ~\citep{nickel2011three}. This model makes use of entity vectors and a relational tensor, where each relation type is modelled as a matrix slice of the tensor. RESCAL is defined as follows:

\begin{equation}
	f_{i, j, k}^{RESCAL} := e_i^TW_ke_j = \sum_{a=1}^{H_e}\sum_{b=1}^{H_e}w_{a,b,k}e_{ia}e_{jb}
\end{equation}

where $f$ is the relational score, $e_i$ is the subject, $e_j$ is the object, and $W$ is the relational tensor. This model linearly models entity-relational interactions by using the dot product operator to construct a latent subject-relation representaiton, before computing a relational score using the dot product operator between the subject-relational representation and the object. RESCAL is thus a linear factorisation composition. \newline
A natural extension to this method is to include a nonlinearity for entity-relational interaction modelling. This extension was introduced by (Jenatton et al. 2012) ~\citep{jenatton2012latent}.This model uses as sigmoid nonlinearity to compute a probability of relational plausibility from the relational score computed using the bilinear tensor product. The model is defined as follows:

\begin{subequations}
	\begin{gather}
		n_{ik}^{j} = e_i^TW_ke_j \\
		\sigma(t) = \frac{1}{1 + e^-t} \\
		\mathbb{P}\left [ R_j(S_i, O_k) = 1 \right ] = \sigma(n_{ik}^{j})
	\end{gather}
\end{subequations}

where $n$ is the relational score, $\sigma$ is the sigmoid function, a value in the range $\in \left ( 0, 1 \right ]$, and $\mathbb{P}$ is the probability of relational plausibility between $S$ and $O$ indexed by relation $R$. This model introduces non-linear entity-relational interaction modelling to tensor factorisation. 

\subsection{Recursive Neural Tensor Factorisation}
Socher and Chen et al (2013) extend nonlinear tensor factorisation by introducing recursive entity representations in the composition of the relational score ~\citep{socher2013reasoning}. Recursive networks (RCN) try to capture the rules for word combinations by constructing compositional representations of two words ~\citep{socher2012semantic}. The NTN model tries to take advantage of these compositional rules by adding them to the bilinear tensor product and augmenting the nonlinear tensor factorisation. The RCN extended nonlinear tensor factorisation model is defined as follows:

\begin{equation}
	g(e_1, R, e_2) =  u_R^Tf(e_1^TW_R^{\left [1:k\right ]}e_2 + V_R\left [ \begin{matrix} e_1 \\ e_2 \end{matrix} \right ] + b_R)
\end{equation}

where $g$ is the relational score, $f$ is the hyperbolic sigmoid function, a value in the range $\in \left [ -1, 1 \right ]$, $e_1^TW_R^{\left [1:k\right ]}e_2 $ is the bilinear tensor product, $V_R\left [ \begin{matrix} e_1 \\ e_2 \end{matrix} \right ]$ is the recursive composition of the subject and object, and $b$ is the bias. \newline

The contrastive max-margin objective is minimised during training. This objective computes a confidence magnitude on a true correct sample - a fact present in the KB, and a confidence magnitude on a corrupt sample - a randomly generated fact not present in the KB. The correct and corrupt samples are used by the objective as follows:

\begin{equation}
	J(\si{\ohm}) =  \sum_{i=1}^N\sum_{c=1}^Cmax(0,1 - g(T^{(i)}) + g(T_c^{(i)})) + \lambda\left\lVert \si{\ohm} \right\rVert_2^2
\end{equation}

where $J$ is the loss value, $\si{\ohm}$ are the model trainable parameters $u, W, V, b, and E$, $N$ is the number of training triplets, $C$ is the number of randomly corrupted facts. $g(T^{(i)}$ is the confidence in the true fact computed by the model, and $ g(T_c^{(i)}$ is the confidence in the corrupt fact computed by the model. Finally $\lambda\left\lVert \si{\ohm} \right\rVert_2^2$ is the ridge ($L_2$) regression regulariser. \bigskip

The NTN model training algorithm also makes use of pre-trained word vectors developed by (Turian et al. 2010) ~\citep{turian2010word}. These are 100-dimensional embeddings, which are aggregated by the set of words that represent an entity. For example for the entity \textit{homo sapien}, the resulting word representation is $V_{homo \; sapiens} = 0.5(V_{homo} + V_{sapien})$. These word vectors are trainable, and updated during back propagation, producing a distributed word representation that aligns with the KB.

\subsection{Modern Optimisation}

Deep model training loops can suffer from noise such as data subsampling, and regularisation techniques such as ridge regression and dropout. Adaptive moment estimation (Adam) is a first order gradient-based stochastic optimisation algorithm that compensates for this this noise ~\citep{kingma2014adam}. Adam computes the first and second order gradients of the loss and add them to the loss, combining the advantages of AdaGrad ~\citep{duchi2011adaptive} and RMSprop ~\citep{tieleman2012lecture}. Adam efficiently regulates parameter update magnitude size, accelerating toward local optima and remaining small in sparse regions. \newpage
Setting model hyperparameters remains a challenge in deep model training. A simple method proposed to address this problem is hyperparameter random search ~\citep{bergstra2012random}. This approach improves on grid search by matching or exceeding model performance in a fraction of the compute time. Hyperparameter random search takes advantage of the fact that for most datasets only a subset of the hyperparameters contribute meaningful variance in model performance, eliminating the need for an exhaustive brute force search over a large combinatorial search space.

\subsection{Training Algorithm}

(Doss et al. 2015) reimplement the NTN model in Tensorflow ~\citep{abadi2016tensorflow}. This model severely under performs the original model, relying on AdaGrad optmisation and the same hyperparaemeters. We apply Adam optimsation as well as hyperparameter optimisation using random search, in attempt to improve its performance. The new training algorithm is as follows: \bigbreak

\begin{algorithm}[H]
	\textbf{loop} // repeat for $N$ experiments using random uniform hyperparameter configuration \\
		\SetAlgoLined
		\textbf{Input} 
		Training set \begin{math} D = \{(x_i, y_i)\}_{i=1}^N \end{math}, samples and labels\;
  		\begin{math} S_{batch} \gets sample(S, b) \end{math} // sample a minibatch of size \begin{math} b \end{math} \\
	 	\For{(x, y) \begin{math} \in S_{batch} \end{math}}{
     			\begin{math} y_{correct} \gets predict(x, \; y_{correct}) \end{math} // predict label for sample \\
			\begin{math} y_{corrupt} \gets predict(x, \; y_{corrupt}) \end{math} // predict label for sample \\
			\begin{math} e \gets contrast(y_{correct}, \; y_{corrupt}) \end{math} // compute error
     			}
		Update model w.r.t. \begin{math}  e \end{math} // using Adam optimser \\
	\textbf{end loop}
	\caption{Updated NTN Training Algorithm}
\end{algorithm} \newpage


% **************************** Section 2 **************************

\section{Hypernetwork Tensor Factorisation}

\subsection{Convolutional Tensor Factorisation}
ConvE introduced the convolutional operator to link prediction model composition ~\citep{dettmers2018convolutional}. Specifically, this operator increases expressiveness in subject-predicate interaction modelling by using 2-dimensional instead of 1-dimensional convolutions, being particularly effective at modelling knowledge graph (KG) nodes with high indegree. ConvE concatenates subject and predicate vectors, creating a 2-dimensional representation. Convolutional filters are then applied to this representation before it is flattened by a fully connected layer. The dot product of this generated representation is then taken with the object vector, producing a measure of confidence in the relation, before a logistic sigmoid is applied to compute a probability of plausibility. The model is defined as follows:

\begin{equation}
	\psi_r(e_s,e_o) = g(vec(f(\left [ \overline{e_s}; \overline{r_r} \right ]*w))W)e_o
\end{equation}

where $psi_r$ is the relational score, $e_s$ is the subject, $e_r$ is the object, and $r_r$ is the predicate. $f$ is the concatenation operation between the subject and predicate, $w$ are the 2-dimensional convolutional filters and $W$ is the parameterised matrix of the fully connected layer. $g$ is a ReLU non-linearity applied to the fully connected layer output. ConvE is thus a convolutional factorisation composition. The computed relational probability is defined as follows: 

\begin{equation}
	p = \sigma(\psi_r(e_s,e_o)) 
\end{equation}

The binary cross-entropy objective is minimised during training. This objective is particularly effective as during we expect only a single true class during inference, modelled as $1$, and every other class to be false, $0$. The objective is defined follows:

\begin{equation}
	L(p, \; t) =  -\frac{1}{N}\sum_i(t \cdot log(p_i) + (1 - t_i) \cdot log(1 - p_i))
\end{equation}

where $L$ is the loss value, $p$ is the relational probability computed by the model, and $t$ is the target. $N$ is the number of entities, and $i$ is the batch sample number. \newpage


\subsection{Hypernetwork Tensor Factorisation}

Hypernetworks are meta networks that generate parameters for a main network ~\citep{ha2016hypernetworks}. The networks essentially index the parameters of a main network as a configuration given input. This input is typically in the form of an embedding vector that describes the entire weights of a given layer. This configuration is learned given a scenario experienced by the main network, for example when performing sequence prediction, it may be advantageous for the main network to change its behaviour (parameter configuration) depending on the sequence window. The hypernetwoirk model can be defined as follows: 

\begin{equation}
	K^j = g(z^j), \quad \forall j = 1, \dots, D
\end{equation}

where $K$ are the layer parameters, $g$ is the hypernetwork composition, and $z$ is the hypernetwork input. \bigskip

HypER is inspired by ConvE, and implements a convolutional operator that models entity-relational interactions ~\citep{balazevic2019hypernetwork}. HypER makes use of a hypernetwork to generate relation-specific convolutional filter. The subject and relational-filter are then used in a convolution operation to generate latent representation which is then flattened by a fully connected layer. The dot product of this generated representation is then taken with the object vector, producing a measure of confidence in the relation, before a logistic sigmoid is applied to compute a probability of plausibility. The model is defined as follows: \newline

% Specifically, this operator increases expressiveness in subject-relation interaction modelling by using grouped convolutions . \newline 
% Grouped convolutions have the benefit ... \newline 
% This model makes use of entity vectors and a relational tensor, where each relation type is modelled as a matrix slice of the tensor.

Socher and Chen et al (2013) extend nonlinear tensor factorisation by introducing recursive entity representations in the composition of the relational score ~\citep{socher2013reasoning}. Recursive networks (RCN) try to capture the rules for word combinations by constructing compositional representations of two words ~\citep{socher2012semantic}. The NTN model tries to take advantage of these compositional rules by adding them to the bilinear tensor product and augmenting the nonlinear tensor factorisation. The RCN extended nonlinear tensor factorisation model is defined as follows:

\begin{equation}
	g(e_1, R, e_2) =  u_R^Tf(e_1^TW_R^{\left [1:k\right ]}e_2 + V_R\left [ \begin{matrix} e_1 \\ e_2 \end{matrix} \right ] + b_R)
\end{equation}

where $g$ is the relational score, $f$ is the hyperbolic sigmoid function, a value in the range $\in \left [ -1, 1 \right ]$, $e_1^TW_R^{\left [1:k\right ]}e_2 $ is the bilinear tensor product, $V_R\left [ \begin{matrix} e_1 \\ e_2 \end{matrix} \right ]$ is the recursive composition of the subject and object, and $b$ is the bias. \newline

The contrastive max-margin objective is minimised during training. This objective computes a confidence magnitude on a true correct sample - a fact present in the KB, and a confidence magnitude on a corrupt sample - a randomly generated fact not present in the KB. The correct and corrupt samples are used by the objective as follows:

\begin{equation}
	J(\si{\ohm}) =  \sum_{i=1}^N\sum_{c=1}^Cmax(0,1 - g(T^{(i)}) + g(T_c^{(i)})) + \lambda\left\lVert \si{\ohm} \right\rVert_2^2
\end{equation}

where $J$ is the loss value, $\si{\ohm}$ are the model trainable parameters $u, W, V, b, and E$, $N$ is the number of training triplets, $C$ is the number of randomly corrupted facts. $g(T^{(i)}$ is the confidence in the true fact computed by the model, and $ g(T_c^{(i)}$ is the confidence in the corrupt fact computed by the model. Finally $\lambda\left\lVert \si{\ohm} \right\rVert_2^2$ is the ridge ($L_2$) regression regulariser. \bigskip

The NTN model training algorithm also makes use of pre-trained word vectors developed by (Turian et al. 2010) ~\citep{turian2010word}. These are 100-dimensional embeddings, which are aggregated by the set of words that represent an entity. For example for the entity \textit{homo sapien}, the resulting word representation is $V_{homo \; sapiens} = 0.5(V_{homo} + V_{sapien})$. These word vectors are trainable, and updated during back propagation, producing a distributed word representation that aligns with the KB.

\subsection{Modern Optimisation}

Deep model training loops can suffer from noise such as data subsampling, and regularisation techniques such as ridge regression and dropout. Adaptive moment estimation (Adam) is a first order gradient-based stochastic optimisation algorithm that compensates for this this noise ~\citep{kingma2014adam}. Adam computes the first and second order gradients of the loss and add them to the loss, combining the advantages of AdaGrad ~\citep{duchi2011adaptive} and RMSprop ~\citep{tieleman2012lecture}. Adam efficiently regulates parameter update magnitude size, accelerating toward local optima and remaining small in sparse regions. \newpage
Setting model hyperparameters remains a challenge in deep model training. A simple method proposed to address this problem is hyperparameter random search ~\citep{bergstra2012random}. This approach improves on grid search by matching or exceeding model performance in a fraction of the compute time. Hyperparameter random search takes advantage of the fact that for most datasets only a subset of the hyperparameters contribute meaningful variance in model performance, eliminating the need for an exhaustive brute force search over a large combinatorial search space.

\subsection{Training Algorithm}

(Doss et al. 2015) reimplement the NTN model in Tensorflow ~\citep{abadi2016tensorflow}. This model severely under performs the original model, relying on AdaGrad optmisation and the same hyperparaemeters. We apply Adam optimsation as well as hyperparameter optimisation using random search, in attempt to improve its performance. The new training algorithm is as follows: \bigbreak

\begin{algorithm}[H]
	\textbf{loop} // repeat for $N$ experiments using random uniform hyperparameter configuration \\
		\SetAlgoLined
		\textbf{Input} 
		Training set \begin{math} D = \{(x_i, y_i)\}_{i=1}^N \end{math}, samples and labels\;
  		\begin{math} S_{batch} \gets sample(S, b) \end{math} // sample a minibatch of size \begin{math} b \end{math} \\
	 	\For{(x, y) \begin{math} \in S_{batch} \end{math}}{
     			\begin{math} y_{correct} \gets predict(x, \; y_{correct}) \end{math} // predict label for sample \\
			\begin{math} y_{corrupt} \gets predict(x, \; y_{corrupt}) \end{math} // predict label for sample \\
			\begin{math} e \gets contrast(y_{correct}, \; y_{corrupt}) \end{math} // compute error
     			}
		Update model w.r.t. \begin{math}  e \end{math} // using Adam optimser \\
	\textbf{end loop}
	\caption{Updated NTN Training Algorithm}
\end{algorithm} \bigbreak
