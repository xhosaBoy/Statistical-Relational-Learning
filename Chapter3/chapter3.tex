%!TEX root = ../thesis.tex
%*******************************************************************************
%****************************** Third Chapter **********************************
%*******************************************************************************
\chapter{Neural tensor factorisation}

% **************************** Define Graphics Path **************************

\ifpdf
    \graphicspath{{Chapter3/Figs/Raster/}{Chapter3/Figs/PDF/}{Chapter3/Figs/}}
\else
    \graphicspath{{Chapter3/Figs/Vector/}{Chapter3/Figs/}}
\fi


Open domain question answering is an area of research concerned with reasoning about knowledge expressed in natural language. Knowledge base (KB) link prediction using latent feature modelling is an approach that has been applied to this task, with varying degrees of success  \unskip ~\citep{nguyen2017novel, diefenbach2018wdaqua, kristiadi2019incorporating}. This approach computes unnormalised scores for potential relationships between entities, and then uses the sigmoid function to produce a probability of plausibility in the interval $ (0, 1)$ for each relational score. These probabilities are then sorted in descending order to rank potential relationships with the highest confidence. \par

\noindent Previous latent feature models have focused on limiting the number of parameters, at the expense of complex entity-relational interaction modelling, in order to scale to large KBs. The neural tensor network (NTN), an extension of tensor factorisation, was the first successful approach to link prediction at taking advantage of expressiveness of deep models. It relies on adding a recursive network (RCN) \unskip ~\citep{pollack1990recursive} representation to the bilinear tensor product, and then using a fully connected layer to compute a relational score, an unnormalised measure of confidence of a potential relationship between two entities. \par

\noindent Convolutional entity representations (ConvE) were similarly effective at extending tensor factorisation approaches, by applying the convolutional operation in modelling subject entity-relational interactions, and then using the multiplication operator between the generated representation and object entity, to compute a relational score. Hypernetworks were then used to extend the ConvE model (HypER) to add further expressive power, by performing a nonlinear transformation of relational filter representations, which are then used for entity-relational interaction modelling. \newpage

\noindent In this chapter we examine the NTN model, and attempt a simple improvement by applying adaptive moment estimation (Adam) optimisation and hyperparameter random search to the training algorithm. We then introduce HypER+, an extension of the HypER model which compensates for covariate shift introduced by nonlinear transformations of relational filters. Finally, we extend HypER+ to make use of pre-trained GloVe word embeddings, and address the problem of entity interaction sparsity in KBs.


% **************************** First Section **************************

\section{Neural Tensor Networks}

\subsection{Neural Tensor Factorisation}
RESCAL introduced the bilinear tensor product for relational scoring \unskip ~\citep{nickel2011three}. This model makes use of entity vectors and a relational tensor, where each relation type is modelled as a matrix slice of the tensor. RESCAL is defined as follows:
\begin{equation}
	f_{i, j, k}^{RESCAL} := e_i^TW_ke_j = \sum_{a=1}^{H_e}\sum_{b=1}^{H_e}w_{a,b,k}e_{ia}e_{jb}
\end{equation}

\noindent where $f$ is the relational score, $e_i$ is the subject, $e_j$ is the object, and $W$ is the relational tensor. This model linearly models entity-relational interactions by using the dot product operator to construct a latent subject-relation representaiton, before computing a relational score using the dot product operator between the subject-relational representation and the object. RESCAL is thus a linear factorisation composition. \par

\noindent A natural extension to this method is to include a nonlinearity for entity-relational interaction modelling. This extension was introduced by (Jenatton et al. 2012) \unskip ~\citep{jenatton2012latent}.This model uses as sigmoid nonlinearity to compute a probability of relational plausibility from the relational score computed using the bilinear tensor product. The model is defined as follows:
\begin{subequations}
	\begin{gather}
		n_{ik}^{j} = e_i^TW_ke_j \\
		\sigma(t) = \frac{1}{1 + e^-t} \\
		\mathbb{P}\left [ R_j(S_i, O_k) = 1 \right ] = \sigma(n_{ik}^{j})
	\end{gather}
\end{subequations}

\noindent where $n$ is the relational score, $\sigma$ is the sigmoid function, a value in the range $\in \left ( 0, 1 \right ]$, and $\mathbb{P}$ is the probability of relational plausibility between $S$ and $O$ indexed by relation $R$. This model introduces non-linear entity-relational interaction modelling to tensor factorisation. 

\subsection{Recursive Neural Tensor Factorisation}
Socher and Chen et al (2013) extend nonlinear tensor factorisation by introducing recursive entity representations in the composition of the relational score \unskip ~\citep{socher2013reasoning}. Recursive networks (RCN) try to capture the rules for word combinations by constructing compositional representations of two words \unskip ~\citep{socher2012semantic}. The NTN model tries to take advantage of these compositional rules by adding them to the bilinear tensor product and augmenting the nonlinear tensor factorisation. The RCN extended nonlinear tensor factorisation model is defined as follows:
\begin{equation}
	g(e_1, R, e_2) =  u_R^Tf(e_1^TW_R^{\left [1:k\right ]}e_2 + V_R\left [ \begin{matrix} e_1 \\ e_2 \end{matrix} \right ] + b_R)
\end{equation}

\noindent where $g$ is the relational score, $f$ is the hyperbolic sigmoid function, a value in the range $\in \left [ -1, 1 \right ]$, $e_1^TW_R^{\left [1:k\right ]}e_2 $ is the bilinear tensor product, $V_R\left [ \begin{matrix} e_1 \\ e_2 \end{matrix} \right ]$ is the recursive composition of the subject and object, and $b$ is the bias. \par

\noindent The contrastive max-margin objective is minimised during training. This objective computes a confidence magnitude on a true correct sample - a fact present in the KB, and a confidence magnitude on a corrupt sample - a randomly generated fact not present in the KB. The correct and corrupt samples are used by the objective as follows:
\begin{equation}
	J(\si{\ohm}) =  \sum_{i=1}^N\sum_{c=1}^Cmax(0,1 - g(T^{(i)}) + g(T_c^{(i)})) + \lambda\left\lVert \si{\ohm} \right\rVert_2^2
\end{equation}

\noindent where $J$ is the loss value, $\si{\ohm}$ are the model trainable parameters $u, W, V, b, and E$, $N$ is the number of training triplets, $C$ is the number of randomly corrupted facts. $g(T^{(i)}$ is the confidence in the true fact computed by the model, and $ g(T_c^{(i)}$ is the confidence in the corrupt fact computed by the model. Finally $\lambda\left\lVert \si{\ohm} \right\rVert_2^2$ is the ridge ($L_2$) regression regulariser. \par

\noindent The NTN model training algorithm also makes use of pre-trained word vectors developed by (Turian et al. 2010) ~\citep{turian2010word}. These are 100-dimensional embeddings, which are aggregated by the set of words that represent an entity. For example for the entity \textit{homo sapien}, the resulting word representation is $V_{homo \; sapiens} = 0.5(V_{homo} + V_{sapien})$. These word vectors are trainable, and updated using backpropagation, producing a distributed word representation that aligns with the KB. \par

\noindent \textbf{Modern Optimisation.} Deep model training loops can suffer from noise such as data subsampling, and regularisation techniques such as ridge regression and dropout. Adaptive moment estimation (Adam) is a first order gradient-based stochastic optimisation algorithm that compensates for this this noise \unskip ~\citep{kingma2014adam}. Adam computes the first and second order gradients of the loss and add them to the loss, combining the advantages of AdaGrad ~\citep{duchi2011adaptive} and RMSprop ~\citep{tieleman2012lecture}. Adam efficiently regulates parameter update magnitude size, accelerating toward local optima and remaining small in sparse regions. \par

\noindent Setting model hyperparameters remains a challenge in deep model training. A simple method proposed to address this problem is hyperparameter random search \unskip ~\citep{bergstra2012random}. This approach improves on grid search by matching or exceeding model performance in a fraction of the compute time. Hyperparameter random search takes advantage of the fact that for most datasets only a subset of the hyperparameters contribute meaningful variance in model performance, eliminating the need for an exhaustive brute force search over a large combinatorial search space. \par

\noindent \textbf{Training Algorithm.} (Doss et al. 2015) reimplement the NTN model in Tensorflow \unskip ~\citep{abadi2016tensorflow}. This model severely under performs the original model, relying on AdaGrad optmisation and the same hyperparaemeters. We apply Adam optimsation as well as hyperparameter optimisation using random search, in attempt to improve its performance. The new training algorithm is as follows: 

\medskip

\begin{algorithm}[H]
	\textbf{loop} // repeat for $N$ experiments using random uniform hyperparameter configuration \\
		\SetAlgoLined
		\textbf{Input} 
		Training set \begin{math} D = \{(x_i, y_i)\}_{i=1}^N \end{math}, samples and labels\;
  		\begin{math} S_{batch} \gets sample(S, b) \end{math} // sample a minibatch of size \begin{math} b \end{math} \\
	 	\For{(x, y) \begin{math} \in S_{batch} \end{math}}{
     			\begin{math} y_{correct} \gets predict(x, \; y_{correct}) \end{math} // predict label for sample \\
			\begin{math} y_{corrupt} \gets predict(x, \; y_{corrupt}) \end{math} // predict label for sample \\
			\begin{math} e \gets contrast(y_{correct}, \; y_{corrupt}) \end{math} // compute error
     			}
		Update model w.r.t. \begin{math}  e \end{math} // using Adam optimser \\
	\textbf{end loop}
	\caption{Updated NTN Training Algorithm}
\end{algorithm} 


% **************************** Second Section **************************

\section{Hypernetwork Tensor Factorisation}

\subsection{Convolutional Factorisation}
ConvE introduced the convolutional operator to link prediction model composition ~\citep{dettmers2018convolutional}. Specifically, this operator increases expressiveness in subject-predicate interaction modelling by using 2-dimensional instead of 1-dimensional convolutions, being particularly effective at modelling knowledge graph (KG) nodes with high indegree. ConvE concatenates subject and predicate vectors, creating a 2-dimensional representation. Convolutional filters are then applied to this representation before it is flattened by a fully connected layer. The dot product of this generated representation is then taken with the object vector, producing a measure of confidence in the relation, before a logistic sigmoid is applied to compute a probability of plausibility. The model is defined as follows:

\begin{equation}
	\psi_r(e_s, \; e_o) = g(vec(f(\left [ \overline{e_s}; \overline{r_r} \right ]*w))W)e_o
\end{equation}

where $\psi_r$ is the relational score, $e_s$ is the subject, $e_r$ is the object, and $r_r$ is the predicate. $f$ is the concatenation operation between the subject and predicate, $w$ are the 2-dimensional convolutional filters and $W$ is the parameterised matrix of the fully connected layer. $g$ is a ReLU non-linearity applied to the fully connected layer output. ConvE is thus a convolutional factorisation composition. The computed relational probability is defined as follows: 

\begin{equation}
	p = \sigma(\psi_r(e_s,e_o)) 
\end{equation}

The binary cross-entropy objective is minimised during training. This objective is particularly effective as during we expect only a single true class during inference, modelled as $1$, and every other class to be false, $0$. The objective is defined follows:

\begin{equation}
	L(p, \; t) =  -\frac{1}{N}\sum_i(t \cdot log(p_i) + (1 - t_i) \cdot log(1 - p_i))
\end{equation}

where $L$ is the loss value, $p$ is the relational probability computed by the model, and $t$ is the target. $N$ is the number of entities, and $i$ is the batch sample number. \newpage

\subsection{Hypernetwork Factorisation}

Hypernetworks are meta networks that generate parameters for a main network ~\citep{ha2016hypernetworks}. The networks essentially index the parameters of a main network as a configuration given input. This input is typically in the form of an embedding vector that describes the entire weights of a given layer. This configuration is learned given a scenario experienced by the main network, for example when performing sequence prediction, it may be advantageous for the main network to change its behaviour (parameter configuration) depending on the sequence window. The hypernetwoirk model can be defined as follows: 

\begin{equation}
	K^j = g(z^j), \quad \forall j = 1, \dots, D
\end{equation}

where $K$ are the layer parameters, $g$ is the hypernetwork composition, and $z$ is the hypernetwork input. \bigskip

HypER is inspired by ConvE, and implements a convolutional operator that models entity-relational interactions ~\citep{balazevic2019hypernetwork}. HypER makes use of a hypernetwork to generate relation-specific convolutional filter. The subject and relational-filter are then used in a convolution operation to generate latent representation which is then flattened by a fully connected layer. The dot product of this generated representation is then taken with the object vector, producing a measure of confidence in the relation, before a logistic sigmoid is applied to compute a probability of plausibility. The model is defined as follows: 

\begin{equation}
	\phi_r(e_1, \; e_2) = f(vec(e_1 * (vec^{-1}(w_rH)))W)e_2
\end{equation}

where $\phi_r$ is the relational score, $e_1$ is the subject, and $e_2$ is the object. $vec^{-1}$ is the transformation that reshapes the output of the hypernetwork into a set of relation-specific convolutional filters, $w$ is the predicate input and $H$ is the parameterised matrix of the fully connected layer of the hypernetwork. $f$ is a ReLU non-linearity applied to the fully connected layer output of the main network. HypER is thus a hypernetwork factorisation composition. The computed relational probability is defined as follows: 

\begin{equation}
	p = \sigma(\phi_r(e_1,e_2)) 
\end{equation}

The HypER training algorithm minimises the binary cross-entropy objective. HypER entities and relations are intialised using Xavier initialisation ~\citep{glorot2010understanding}, drawn from a distribution with zero mean and finite variance. \newpage

\textbf{Hyper Covariate Shift}. We make the observation that hypernetworks may also suffer from covariate shift. The network parameters are adjusted during training, resulting in a distributional drift across the entire main network as training progresses. To address this problem, we introduce batch normalisation between the hypernetwork and main network. We adjust the model accordingly, and introduce HypER+. \bigskip 

\textbf{Training Algorithm}. HypER+ is trained using the binary cross-entropy objective. We use the same hyperparameters as the original HypER model, as well as the Adam optimiser. The HypER+ training algorithm is as follows: \newline

\begin{algorithm}[H]
	\SetAlgoLined
	\textbf{Input} 
	Training set \begin{math} D = \{(x_i, y_i)\}_{i=1}^N \end{math}, samples and labels\;
  	\begin{math} S_{batch} \gets sample(S, b) \end{math} // sample a minibatch of size \begin{math} b \end{math} \\
	 \For{(x, y) \begin{math} \in S_{batch} \end{math}}{
     		\begin{math} y' \gets predict(x, \; y') \end{math} // predict label for sample \\
		\begin{math} e \gets y' - y \end{math} // compute error
     		}
	Update model w.r.t. \begin{math}  e \end{math} // using Adam optimser \\
	\caption{HypER+ Training Algorithm}
\end{algorithm} \bigskip
 \bigskip
 
\textbf{Pre-trained Word Vectors}. We then apply pre-trained GloVe word vectors ~\citep{pennington2014glove} to the HypER+ training algorithm. We use these word vectors to initialise  200-dimensional embeddings, which are aggregated by the set of words that represent an entity or relation, the same method of aggregation used for pre-trained word vectors used during NTN training. Similarly, these word vectors are trainable, and updated using backpropagation.  


%********************************** %Third Section  **************************************

\section{Summary}

In this chapter we've discussed: \newline
\textbf{Neural Tensor Networks.} An early successful attempt at using the expressive power of neural networks for link prediction. This model also makes use of RCNs, which try to model semantic compositionality, and this model makes use of pre-trained word vectors to improve link prediction performance. We apply modern stochastic and hyperparameter optimsiation techniques to the NTN training algorithm, namely Adam and hyperparameter random search. \newpage
\textbf{Hypernetwork Tensor Factorisation.}  Hypernetworks can be used to generate parameters for other networks. This approach has the effect of applying a configuration to a neural network, and changing its behaviour given indexing input. A hypernetworks were used to generate parameters for a convolutional network in the model HypER. Concretely, HypER generates relation-specific convolutional filters which are used in a neural factorisation approach to link prediction. We adjust HypER to compensate for the covariate shift caused by the hypernetwork and introduce HypER+. We then extend HypER+ to make use of GloVe pre-trained word vectors by initialising entities and relations using these embeddings. \newline
