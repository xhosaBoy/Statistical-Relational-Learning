%!TEX root = ../thesis.tex
%*******************************************************************************
%****************************** Third Chapter **********************************
%*******************************************************************************
\chapter{Neural Tensor Factorisation}

% **************************** Define Graphics Path **************************

\ifpdf
    \graphicspath{{Chapter3/Figs/Raster/}{Chapter3/Figs/PDF/}{Chapter3/Figs/}}
\else
    \graphicspath{{Chapter3/Figs/Vector/}{Chapter3/Figs/}}
\fi


Open domain question answering is an area of research concerned with reasoning about knowledge expressed in natural language. Knowledge base (KB) link prediction using latent feature modelling is an approach that has been applied to this task, with varying degrees of success  ~\citep{nguyen2017novel, diefenbach2018wdaqua, kristiadi2019incorporating}. This approach aims to rank plausible relationships between entities by using using trainable embedding representaitons. Previous latent feature models have focused on limiting the number of parameters so as to be able to scale to large KBs, at a cost of complex entity-relational interaction modelling. \newline
Neural tensor networks (NTN), An extension to tensor factorisation was the first successful approach at taking advantage of expressiveness of deep models. It relied on adding a recursive network (RCN) representation to the bilnear tensor product, and wrapping that representation in a fully connected layer to then compute a relational score. Convolutional entity representations (ConvE) were similarly effective at extending tensor factorisation approaches by applying the convolutional operation in modelling entity-relational interactions, before completing a factorisation to compute a relational score. Hypernetworks were then applied to the ConvE model (HypER) to add further expressive power by pre-computing relational filter representations for subsequent use in a neural tensor factorisation. \newline
In this chapter we examine the NTN model, and attempt a simple improvement by applying early stopping and hyperparameter random search to the training algorithm. We then introduce HypER+, an extension to the HypER model which compensates covariate shift introduced by augmenting relational filters with a hypernetwork. We then extend HypER+ to make use of pre-trained GloVe word embeddings, and address the problem of entity interaction sparsity in KBs.


% **************************** Section 1 **************************

\section{Neural Tensor Networks}

\subsection{Neural Tensor Factorisation}
RESCAL introduced the bilinear tensor product for relational scoring ~\citep{nickel2011three}. This model makes use of entity vectors and a relational tensor, where each relation type is modelled as a matrix slice of the tensor. RESCAL is defined as follows:

\begin{equation}
	f_{i, j, k}^{RESCAL} := e_i^TW_ke_j = \sum_{a=1}^{H_e}\sum_{b=1}^{H_e}w_{a,b,k}e_{ia}e_{jb}
\end{equation}

where $f$ is the relational score, $e_i$ is the subject, $e_j$ is the object, and $W$ is the relational tensor. This model linearly models entity-relational interactions by using the dot product operator to construct a latent subject-relation representaiton, before computing a relational score using the dot product operator between the subject-relational representation and the object. RESCAL is thus a linear factorisation composition. \newline
A natural extension to this method is to include a nonlinearity for entity-relational interaction modelling. This extension was introduced by (Jenatton et al. 2012) ~\citep{jenatton2012latent}.This model uses as sigmoid nonlinearity to compute a probability of relational plausibility from the relational score computed using the bilinear tensor product. The model is defined as follows:

\begin{subequations}
	\begin{gather}
		n_{ik}^{j} = e_i^TW_ke_j \\
		\sigma(t) = \frac{1}{1 + e^-t} \\
		\mathbb{P}\left [ R_j(S_i, O_k) = 1 \right ] = \sigma(n_{ik}^{j})
	\end{gather}
\end{subequations}

where $n$ is the relational score, $\sigma$ is the sigmoid function, a value in the range $\in \left ( 0, 1 \right ]$, and $\mathbb{P}$ is the probability of relational plausibility between $S$ and $O$ indexed by relation $R$. This model introduces non-linear entity-relational interaction modelling to tensor factorisation. 

\subsection{Recursive Neural Tensor Factorisation}
Socher and Chen et al (2013) extend nonlinear tensor factorisation by introducing recursive entity representations in the composition of the relational score ~\citep{socher2013reasoning}. Recursive networks (RCN) try to capture the rules for word combinations by constructing compositional representations of two words ~\citep{socher2012semantic}. The NTN model tries to take advantage of these compositional rules by adding them to the bilinear tensor product and augmenting the nonlinear tensor factorisation. The RCN extended nonlinear tensor factorisation model is defined as follows:

\begin{equation}
	g(e_1, R, e_2) =  u_R^Tf(e_1^TW_R^{\left [1:k\right ]}e_2 + V_R\left [ \begin{matrix} e_1 \\ e_2 \end{matrix} \right ] + b_R)
\end{equation}

where $g$ is the relational score, $f$ is the hyperbolic sigmoid function, a value in the range $\in \left [ -1, 1 \right ]$, $e_1^TW_R^{\left [1:k\right ]}e_2 $ is the bilinear tensor product, $V_R\left [ \begin{matrix} e_1 \\ e_2 \end{matrix} \right ]$ is the recursive composition of the subject and object, and $b$ is the bias. \newline

The NTN model training algorithm also makes use of pre-trained word vectors developed by (Turian et al. 2010) ~\citep{turian2010word}. These are 100-dimensional embeddings, which are aggregated by the set of words that represent an entity. For example for the entity \textit{homo sapien}, the resulting word representation is $V_{homo \; sapiens} = 0.5(V_{homo} + V_{sapien})$. These word vectors are trainable, and updated during back propagation, producing a distributed word representation that aligns with the knowledge base.

\subsection{Early Stopping and Random Search}



\subsection{Training Algorithm}


% **************************** Section 2 **************************

\section{Hypernetwork Tensor Factorisation}

