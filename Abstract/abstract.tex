% ************************** Thesis Abstract *****************************
% Use `abstract' as an option in the document class to print only the titlepage and the abstract.
\begin{abstract}
Reasoning over knowledge expressed in natural language is a problem at the forefront of artificial intelligence. Question answering is one of the core tasks of this problem, and is concerned with giving machines the capability of generating an answer given a question, by mimicking the step-by-step reasoning behaviour of humans. Relational learning, in combination with information retrieval, has been explored as a framework for solving this problem. Knowledge graphs (KGs) are used to represent facts about multiple domains as entities (nodes) and relations (edges), and the Resource Description Framework formalism, subject-predictate-object, is used to encode these facts. Link prediction then powers knowledge discovery by scoring possible relationships between entities. \newline
\noindent This thesis explores latent feature modelling using tensor factorisation as an approach to link prediction. Tensor decompositions are an attractive approach as relational domains are usually high-dimensional and sparse, a setting where factorisation methods have shown very good results. Previous approaches have focused on shallow models that can scale to large datasets, and recently deep models have been applied, neural tensor factorisation, as these models are more expressive, and automatically learn the most useful latent features for entities and relations. In this work we introduce training algorithm optimisations to the Neural Tensor Network (NTN) and HypER neural tensor factorisation models. \newline
\noindent  We make use of the TensorFlow reimplimentation of NTNs and apply early stopping, adaptive moment estimation and random search hyperparamter optimisation. We see improvements in both cost and accuracy over the baseline reimplementation using standard link prediction benchmark datasets,  Wordnet and Freebase. We then apply optimisations to the HypER model training algorithm, beginning with compensating for covariate shift caused by hypernetworks by using batch normalisation, and introduce HypER+. We see similar performance to the HypER baseline using the WN18 dataset, and see significant improvement using the FB15k dataset. We then extend our optimisation by initialising entities and relations using GloVe pre-trained word vectors. We see marginal improvements over the HypER baseline using the WN18RR and FB15k-237 datasets. Despite small improvements, our results establish HypER+ as the state-of-the-art in latent feature modelling based link prediction. 
\end{abstract}
 