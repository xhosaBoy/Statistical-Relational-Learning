% ************************** Thesis Abstract *****************************
% Use `abstract' as an option in the document class to print only the titlepage and the abstract.
\begin{abstract}
Reasoning over knowledge expressed in natural language is a problem at the forefront of artificial intelligence (AI). Open domain question answering is a core task of this problem, and link prediction over entities and relations in knowledge graphs has demonstrated potential in solving this problem. This thesis explores latent feature modelling, a class of statistical relational learning, using tensor factorisation as an approach to link prediction. Previous approaches have focused on shallow models that can scale to large datasets; recently deep models have been applied to tensor factorisation, neural tensor factorisation. These models are more expressive, and automatically learn the most useful latent features for entities and relations. In this work, we introduce training algorithm optimisations to the Neural Tensor Network (NTN) and HypER neural tensor factorisation models. We make use of the TensorFlow reimplimentation of NTNs, and apply early stopping, adaptive moment estimation and random search hyperparamter optimisation. We see improvement in both cost and accuracy over the baseline reimplementation using previous standard link prediction benchmark datasets, Wordnet and Freebase. We then apply training algorithm optimisations to the HypER model. We begin by compensating for covariate shift introduced by hypernetwork relational filter generation, and introduce batch normalisation for relations to address this problem, HypER+. We see similar performance in link prediction benchmark metrics over the HypER baseline using the WN18 dataset, and see significant improvement using the FB15k dataset. We then extend our algorithmic optimisation by initialising entities and relations using GloVe pre-trained word vectors. We see marginal improvements over the baseline model using the WN18RR and FB15k-237 datasets. We establish HypER+ as the state-of-the-art model in link prediction using latent feature modelling. 
\end{abstract}
 