% ************************** Thesis Afrikaans Abstract *****************************
% Use `abstract' as an option in the document class to print only the titlepage and the abstract.
\begin{opsomming}
Redenering oor kennis wat in natuurlike taal uitgedruk word, is 'n
probleem aan die voorpunt van kunsmatige intelligensie. Die
beantwoording van vrae is een van die kerntake van hierdie probleem,
en poog om masjiene die vermo\"e te gee om 'n antwoord te skep vir
'n gegewe vraag, deur die redenasiegedrag van mense na te boots.
Verhoudingsleer, in kombinasie met die inwin van inligting, is al
ondersoek as 'n raamwerk vir die oplossing van hierdie probleem.
Kennisgrafieke (KG's) word gebruik om feite oor veelvuldige domeine
as entiteite (punte) en verhoudings (lyne) voor te stel, en die
bronbeskrywingsraamwerk-formalisme, nl.\ onderwerp-predikaat-voorwerp,
word gebruik om sulke feite te enkodeer. Skakelvoorspelling dryf dan
kennisontdekking deur moontlike verhoudings tussen entiteite te bepunt. \newline
\noindent Hierdie tesis ondersoek latente kenmerkmodellering met behulp van
tensorfaktorisering, as 'n benadering tot skakelvoorspelling. Tensor-ontbindings
is 'n aantreklike benadering, aangesien verhoudingsdomeine gewoonlik
hoogdimensioneel en yl is; omstandighede waar faktoriseringsmetodes reeds
baie goeie resultate getoon het. Vorige benaderings het op vlak modelle
gefokus, wat kan skalleer met groot datastelle. Meer onlangs is diep modelle
toegepas, spesifiek neurale tensorfaktoriseringsmodelle, aangesien hierdie
modelle meer ekspressief is en outomaties die nuttigste latente kenmerke vir
entiteite en verhoudings kan aanleer. In hierdie werk stel ons optimering van
afrigalgoritmes voor vir die neurale tensornetwerk (NTN) en HypER neurale tensorfaktoriseringsmodelle. \newline
\noindent Ons maak gebruik van die TensorFlow-herimplementering van NTN's, en pas
vroe\"e-stop, aanpasbare momentskatting, sowel as hiperparameteroptimering
met ewekansige soeke, toe. Ons sien verbeterings in koste sowel as akkuraatheid
oor die basiese NTN-herimplementering, in die standaard
skakelvoorspellingsdatastelle WordNet en Freebase. Ons pas dan optimerings
toe op die HypER-model se afrigtingsalgoritme. Ons begin met die kompensering
van kovariantskuif wat deur hipernetwerke veroorsaak word, met behulp van bondelnormalisering, en stel HypER+ voor. Ons sien prestasies soortgelyk aan die
HypER-basismodel op die WN18-datastel, en beduidende verbetering op die
FB15k-datastel. Ons brei ons optimering uit deur entiteit- en verhoudingsinbeddings
te inisialiseer met vooraf-afgerigte woordvektore van die GloVe-taalmodel. Ons sien
marginale verbeterings oor die basismodel op die WN18RR en FB15k-237
datastelle. Ons resultate vestig HypER+ as 'n mededingende model in latente kenmerkmodelleringsgebaseerde skakelvoorspelling.
\end{opsomming}
 